\section*{Teoretická část}
Při interakci atomu s $\gamma$-zářením může dojít k fotoelektrickému jevu, z elektronového obalu je vyražen elektron a atom zůstane v excitovaném stavu. Při deexcitaci atom vyzařuje rentgenové záření, jehož spektrum je pro daný prvek charakteristické \cite{skripta}. Podle rentgenového spektra můžeme za jistých okolností určit, o jaký prvek se jedná.

K excitaci atomu dochází také při K-záchytu \cite{skripta}.

Mějme slitinu ze dvou prvků A a B, jejichž výtěžky při měření čistého prvku jsou $v_A$ respektive $v_B$. Pro výtěžky $v^s_A$ resp. $v^s_B$ při měření ve sloučenině potom platí
\begin{equation}
\frac{v_A^s}{w_A} \frac{w_B}{v_B^s} = \frac{v_A}{v_B} \,,
\end{equation}
kde $w_A$ a $w_B$ jsou relativní zastoupení prvku A resp. B. Pokud je slitina čistá, platí navíc
\begin{equation}
w_A+w_B=1 \,,
\end{equation}
z čehož vyplývá
\begin{equation} \label{e:konc}
w_A=\frac{v_A^s \cdot v_B}{v_A^s \cdot v_B + v_B^s\cdot v_A} \,, \qquad \qquad 
w_B=\frac{v_B^s \cdot v_A}{v_A^s \cdot v_B + v_B^s\cdot v_A} \,.
\end{equation}
\section*{Diskuze}
Naměřené závislosti výtěžku na protonovém čísle dobře neodpovídá ani lineární ani kvadratický fit. Nicméně je zřejmý rostoucí trend, což souhlasí s \cite{skripta}. Kvadratický fit je samozřejmě pro nízká Z nepoužitelný. Pro interpolaci hodnot v měřeném rozsahu považujeme také lineární fit za lepší.

Pro jiné než K$\upalpha$ přechody jsme závislost nesestavovali, porovnávat výtěžky různých přechodů nemá význam.

Pokusili jsme se měřit ještě spektrum $_{26}$Fe, jenže kvůli nízkému protonovému číslu byl výtěžek příliš nízký pro určení spektrálních čar.

U Pb jsme určili, že se jedná o L přechody, především díky lehce rozpoznatelnému vzhledu olova a tomu, že druhý peak (s vyšší energií) nebyl výrazně slabší než ten první.

Nepřímo změřené relativní podíly u slitin 5 a 13 považujeme za poměrně nepřesné. Nejistotu jsme odhadli s ohledem na 

Mohlo se stát, že některý z peaků spektra vzorku se kryl s některým z peaků zářiče. V tom případě by byl výtěžek nadhodnocený. Stát se to mohlo především v okolí \SI{17.5}{\keV}, kde bylo možné naměřit až o \SI{6}{cps} více. 